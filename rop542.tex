\documentclass{book}
\usepackage[T1]{fontenc}
\usepackage[french]{babel}
\usepackage{amsmath,amsfonts,amsthm,amssymb}

\usepackage{minitoc}

\usepackage{xcolor}

\newtheoremstyle{own}%
    {3pt}% Space above
    {3pt}% Space below
    {}% Body font
    {}% Indent amount
    {\color{blue}\bfseries}% Theorem head font
    {:}% Punctuation after theorem head
    {\newline}% Space after theorem head
    {}% Theorem head spec

\theoremstyle{own}
\newtheorem{example}{Example}[section]

\theoremstyle{definition}
\newtheorem{definition}{Definition}[section]

\theoremstyle{remark}
\newtheorem*{remark}{Remarque}

\newtheorem{prop}{Proposition}[section]

\newtheorem{theorem}{Theorem}[section]

\usepackage{tikz}
\newcommand*\circled[1]{\tikz[baseline=(char.base)]{
            \node[shape=circle,draw,inner sep=2pt] (char) {#1};}}

\renewcommand\qedsymbol{$\blacksquare$}

\newcommand{\todo}[1]{\textcolor{red}{#1}}
\begin{document}

\frontmatter
\dominitoc
\tableofcontents

\todo{Retravailler les commande remarque et definition}\\
\todo{pour les d�finition mettre le mot tout de suite apres begin example suivi d'un line break}\\

\mainmatter
\chapter{Pr�liminaires}
\minitoc

	\section{Optimisation}

		Avant d'entreprendre l'�tude de probl�mes d'optimisation il est bien de d�finir ce qu'est un probl�me d'optimisation.\\

Celui-ci consiste a d�terminer la valeur possible qu'une fonction r�elle $$f\colon E \to \mathbb{R}$$ nomm� fonction objectif, peut prendre dans l'ensemble $E$, nomm� ensemble r�alisable.\\

\begin{itemize}
	\item Pour la minimisation
		$$
			f^* = \inf_{x \in E} f(x)
		$$

		Cela signifie que
		\begin{enumerate}
			\item $f(x) \geq f^* \text{~~~~}\forall x \in E$
			\item $\forall \epsilon > 0, \exists x_{\epsilon} \in E \mid f(x_{\epsilon}) < f^* + \epsilon$
			\begin{example}
				Soit $E=\mathbb{R}$ \\
				Si $f(x) = e^x$, on a que
				$$0 = \inf_{x \in E} f(x), \text{ mais } f(x) > 0 \text{~~} \forall \in E$$

				Par contre si $f(x)=x^2$ on a que\\
				$$0 = f(0) = \inf_{x \in E} f(x) = \min_{x \in E} f(x)$$
				
			\end{example}
		\end{enumerate}

	\item Pour la maximisation
		$$
			f^* = \sup_{x \in E} f(x)
		$$

		\begin{enumerate}
			\item $f(x) \leq f^* \text{~~~~} \forall x \ in E$
			\item $\forall \epsilon > 0, \exists x_{\epsilon} \in E \mid f(x_\epsilon) > f^*-\epsilon $
		\end{enumerate}
\end{itemize}

Notons que maximiser $f$ revient � minimiser $-f$. Ainsi sans perte de g�n�ralit�, on peut uniquement consid�rer les probl�mes de minimisation.\\

Sans ajouter d'hypoth�se sur la fonction $f$ et l'ensemble $E$, il n'est pas certain que l'on puisse trouver un �l�ment $x^*$ tel que $f(x^*) = f^*$. Lorsque c'est le cas, la formulation math�matique devient:

$$
	f(x^*)=f^*=\min_{x\in E} f(x)
$$

De mani�re �vidente, les probl�mes pour lesquels il existe un �l�ment de l'ensemble $E$ tel que $f(x^*)=f^*$ seront particuliairement int�ressants. Un tel pointe de l'ensemble $E$ est nomm� optimum et il peut �tre un minimum ou un maximum.

\section{Types d'optimums}
	Sans hypoth�se additionnelles, les probl�mes tels que d�crits pr�c�demment sont en g�n�ral impossible � r�soudre. Consid�rons la fig. 1

\todo{Ajouter la fig.1}
	\begin{itemize}
		\item O� se situe le minimum de la fonction?
		\item Comment identifier lequel des nombreux (voir infinis!) miminum apparant est le plus petit?
	\end{itemize}

	Pour arriver � �tudier les probl�mes d'optimisation nous allons classifier les optimums selons diff�rents crit�res.

	\subsection{Optimums locaux et globaux}
		Jusqu'� maintenant, nous avons d�fini les optimums en comparant la valeur de la fonction $f$ � l'optimum avec sa valeur en tout autre point de $E$. Ce type de probl�me est connu sous le nom d'optimisation globale. Si on baisse un peu les attentes et que l'on compare les valeurs de la fonction $f$ dans un voisinage, alors $x^*$ est un minimum local s'il existe $$\epsilon > 0 \text{ t.q. } f(x) \geq f(x^*), \text{  }\forall x \in E \cap V_\epsilon(x^*)$$
qui est un voisinage de diam�tre $\epsilon$ centr� en $x^*$.

	\subsection{Optimums stricts}
		La notion d'optimum strict conserne le fait que d'autre point du voisinage ne puissent avoir la m�me valeur que $x^*$. Par exemple � la figure 2, les zones encercl�es contiennent des optimums qui ne sont pas stricts, tandis que les autres le sont.

		\todo{Ajouter la fig. 2}

		\begin{definition}
			Un minimum local $x^*$ est dit strict s'il existe une valeur $\epsilon > 0 \text{ t.q. } \forall x \in V_\epsilon(x^*)$ et pour $x\neq x^*$ $f(x)>f(x^*)$.
		\end{definition}
		\todo{Ajouter underline dans la d�finition}

	\subsection{Optimums isol�s}
		Sur la figure 2, les optimums encercl�s ne sont pas stricts, mais chaque ensemble d'optimums est s�par� les uns des autres. La notion d'optimum isoll� formalise cette situation.

		\begin{definition}
			Un ensemble connexe d'optimums $O$ est dit isol� si quelque soit un optimum $x \not\in O$, la distance de $x$ � $O$ est born� inf�rieurement par une constante positive.
		\end{definition}
		\todo{Ajouter underline dans la d�finition}

		\begin{example}
			La fonction $\left(x sin\left(\frac{1}{x}\right)\right)^2$ comporte une accumulation de minimum locaux � l'origine. L'origine n'est pas un optimum isol� ni strict, car une infinit� de points valent $0$ pr�s de l'origine.
		\end{example}

\section{Conditions d'optimalit�}
\todo{v�rifier l'orthographe du titre de la section}
	Malgr� les classifications pr�c�dentes, il demeure difficile d'identifier ou de v�rifier qu'un point $x^*$ est un optimum d'un probl�me . En effet, les d�finitions dont nous disposons exigeraient de comparer les valeurs d'une fonction en un point $x^*$ avec sa valeur en une infinit� de points voisins de $x^*$. L'analyse math�matique vient alors � notre secours.

	\subsection{Conditions pour un point stationnaire}
		On apprend dans les permiers cours de calcul qu'une fonction poss�de des optimums locaux en des points qui annulent sa d�riv�e. De tels points sont nomm�s point stationnaires. Les points stationnaires pour une fonction r�elle peuvent �tre de trois types: minimums locaux, maximum locaux et points d'inflexions.\\

Par cons�quent, tout minimum local est un point stationnaire, mais l'inverse n'est pas vrai. Les points stationnaires satisfont donc a la condition n�cessaire d'optimimalit� de premier ordre: $f'(x)=0$
	\subsection{Conditions pour un optimum}
		On apprend aussi dans les cours de calcul que si,  en point stationnaire, la d�riv� seconde d'une fonction est positive il s'agit d'un minimum local. Si elle est n�gative, il s'agit d'un maximum local et si elle est nulle one ne peut rien conclure. Les points stationnaires qui satisfons � la condition suffisante d'optimalit� de second ordre: $f''(x)>0$ sont donc des minimums.

		\begin{theorem}
			Soit $f$ une fonction de classe $C^2$. Un point $x^*$ qui satisfait aux conditions n�cessaire et suffisante d'optimalit� est un minimum local strict et isol�.
		\end{theorem}

		Le cas $f''(x^*)=0$ comporte une ambiguit� que l'on peut illustrer par les fonctions $$f_1(x)=x^4 \text{ et } f_2(x)=-x^4$$

		en $x^*=0$, pour les deux fonctions, on a que 
$$f'(0)=f''(0)=0$$

bien que l'origine soit un minimum pour $f_1$ et un maximum pour $f_2$.
\chapter{Programation lin�aire}
\minitoc
\todo{enlever les numero de page, le titre content et les lignes horizontale}

	\section{Mod�lisation math�matique}
		La programation lin�aier est une branche des math�matiques ayant pour objectif la r�alisation par des techniques d'optimisation de nombreux probl�mes �comomique et industriels. La fonction objectif de m�me que les contraintes � respecter doivent d�pendre lin�airement d'un ensemble de variables de d�cision. La solution optimale � un programme lin�aire sera un ensemble de valeurs � donner aux variables de d�cision qui maximisera lou minimisera la valeur de la fonction objectif.
	
	\section{Repr�sention graphique d'in�quation lin�aire � 2 variables}
		Une in�quation repr�sente une in�galit� entre deux quantit�s alg�briques. Cette in�galit� contient des inconnues, et r�soudre une in�quation signifiw trouver une ou des valeurs pour ces inconnues qui rendent vraie l'in�galit�. En particulier, une in�quation lin�aire � deux variables aura une des formes suivantes: 

		$$
			\left.
			\begin{array}{l}
				ax + by < c \\
				ax + by \leq c \\
				ax + by > c \\
				ax + by \geq c \\
			\end{array}
			\right\} , a, b \in \mathbb{R}
		$$

		La droite $ax + by = c$ divise le plant en deux r�gions: l'ensemble des points$(x,y)$ respectant l'in�galit� et l'ensemble des points $(x,y)$ ne la respectant pas.
\\ \todo{FBR ajouter une figure pour ca}

	Dans un programme lin�aire � deux variables chaque contrainte prendra la forme d'une telle in�galit�. En mettant en commun l'ensemble des contraintes on pourra obtenir une r�gion du plan, appel� domaine r�alisable, o� toutes les contraintes sont respect�es.

	\section{Recherche graphique d'un optimum}
		Pour une programme lin�aire � deux variables, le domaine r�alisable prendra la forme d'un polyg�ne convexe appel� polygone de contraintes. Si la valeur optimale d'une fonction objectif d'un probl�me existe, alors cette valeur est atteinte � au moins un des sommets du polygone de contraintes.
\\ \todo{generer et ajouter la figure}

	\section{M�thodes du simplexe}
	\todo{refaire la mise en page et separer les paragraphes}\\
		L'algorithme du simplexe est une m�thode fondamentale pour r�soudre des probl�mes d'optimisation lin�aire. �tant donn� un ensemble d'in�galit� lin�aire sur n variables r�elles l'algorithme permet de minimiser (ou maximiser) une fonction objectif qui est elle aussi lin�raire. La forme standard d'un programme lin�aire est 
		$$
			\begin{array}{rl}
				\text{maximiser} & \left. z=c_1x_1 + c_2x_2 + \dots + c_nx_n\right\}\text{fonction objectif}\\
				\\
				\text{sujet �} & \\
				&  \left.
					\begin{array}{l}
						a_{11}x_{1}+a_{12}x_{2}+ \dots + a_{mn}x_n \leq b_1\\
						a_{21}x_{1}+a_{22}x_{2}+ \dots + a_{mn}x_n \leq b_1\\
						\text{~~~~}\vdots \\
						a_{m1}x_{1}+a_{m2}x_{2}+ \dots + a_{mn}x_n \leq b_1
					\end{array}
					\right\}\text{contraintes}\\
				\\
				 & \text{avec }b_i\geq 0 \text{pour tou t} i
			\end{array}\\
		$$

		Pour amener un programme lin�aire de sa forme standard � sa forme canonique il faut convertir les contraintes en �galit�s. Cela s'effectue en ajoutant ou en retranchant une quantit� non n�gative $e_i$ au terme de gauche selon le type d'in�galit�. On appellera $e_i$ variable d'�cart ou variable de surplus selon le cas.

		\begin{example}
			\todo{ajouter partie1}\\
			\todo{ajouter l'�nonc�}\\
				\underline{Forme standard}\\
				\todo{indenter le bloc de texte}\\
					$x:\text{quantit� de sirop(L)}$\\
					$y:\text{quantit� de tire(L)}$\\
					maximizer $R:8x+20y$\\
					sujet �\\
					$30x+80y \leq 60 000$\\
					$\frac{1}{20}x +\frac{1}{10}y \leq 90$\\
					$y \leq 500$\\
					$x \geq 0$\\
					$y \geq 0$\\
		
				\noindent \underline{Forme canonique}\\
					\todo{indenter le bloc de texte}\\
					maximiser $r:8x+20r$\\
					sujet �\\
					$30x+80y+e_1=60 000$\\
					$\frac{1}{20}x + \frac{1}{2}y+e_2=90$\\
					$y+e_3=500$\\
					$x\geq 0$\\
					$y\geq 0$\\
					$e_1\geq 0$\\
					$e_2\geq 0$\\
					$e_3\geq 0$\\
					$e_4\geq 0$\\
		\end{example}

		\todo{enlever l'indent}\\
		Soit la matrice 
		$$
			\begin{array}{rl}
				A=& \left[ \begin{array}{cccc}
						a_{11} & a_{12} & \dots & a_{1n} \\
						a_{21} & a_{22} & \dots & a_{2n} \\
						\vdots &  & \ddots & \vdots\\
						a_{m1} & a_{m2} & \dots & a_{mn} 
					\end{array} \right]x
			\end{array}
		$$
		et les vecteurs
		$$
			\begin{array}{rl}
				c&=\left(c_1, c_2, \dots, c_n \right)\\
				x&=\left(x_1, x_2, \dots, x_n \right)\\
				b&=\left(b_1, b_2, \dots, b_n \right)
			\end{array}
		$$
		On pout exprimer un programme lin�aire sous la forme canonique suivante:\\
		\todo{refaire le formatage selon ce qui est dans les notes}
		mininiser $z-c^t$ sujet � $Ax=b$ et $x \geq 0$.\\

		Pour une probl�me de maximisation standard l'algorithme du simplexe est compos�e des �tapes suivantes:\\

		\begin{enumerate}
			\item �crire le probl�me de maximisation sous la forme canonique et introduisant des variables d'�cart et construire le tableau du simplexe initial.

			\item La derni�re ligne du tableau su simplexe comporte-t-elle des �l�ments n�gatifs:\\
				\begin{itemize}
					\item Si oui: �tape 3
					\item Sinon: solution optimale
				\end{itemize}
			\item Choisir pour variable entrante celle qui est associ�e � la valeur n�gative la plus grande en valeur absolue dans la derni�re ligne du tableau. La colonne de la variable entrante est la colonne pivot.
			\item Le quotient de la valeur d'une des variables de vase par l'�l�ment correspondant de la colonne pivot est-il positif?
				\begin{itemize}
					\item Si non: le probl�me n'a pas de solution
					\item Si oui: �tape 5
				\end{itemize}

			\item Choisir pour variable sortante celle qui est associ�e au plus faible quotient positif fini de la valeur divis�e par l'�l�ment correspondant de la colonne pivot. La ligne de la variable sortante est la ligne pivot.

			\item D�terminer le pivot: l'�l�ment � l'intersection des lignes et colonnes pivots et effectuer le pivotage pour produire un nouveau tableau du simplexe o� la variable entrante remplace la variable sortante et revenir a l'�tape 2
		\end{enumerate}

		\todo{donner un titre genre �rabli�re partie 2}
		\begin{example}
			\todo{ajuste les = de la col. quotient}
			$$
			\begin{array}{r|cccccc|c|l}
				Base & x & y & e_1 & e_2 & e_3 & R & Valeur & Quotient\\
				\hline
				e_1 & 30 & 80 & 1 & 0 & 0 & 0 & 60000 & \frac{60000}{80} = 750 \\
				e_2 & \frac{1}{20} & \frac{1}{10} & 0 & 1 & 0 & 0 & 90 & \frac{90}{\frac{1}{10}} = 900 \\
				e_3 & 0 & \circled{1} & 0 & 0 & 1 & 0 & 500 & \frac{500}{1} = 500 \\
				\hline
				R & -8 & -20 & 0 & 0 & 0 & 1 & 0
			\end{array}
			$$
			Pivotage:
			$$
				\begin{array}{r|cccccc|c|l}
					Base & x & y & e_1 & e_2 & e_3 & R & Valeur & Quotient\\
					\hline
					e_1 & \circled{30} & 0 & 1 & 0 & -80 & 0 & 20000 & \frac{20000}{30} = 666.\overline{66}\\
					e_2 & \frac{1}{20} & 0 & 0 & 1 & -\frac{1}{10} & 0 & 40 & \frac{40}{\frac{1}{20}} = 800\\
					y & 0 & 1 & 0 & 0 & 1 & 0 & 500 & \frac{500}{0} = \text{nope, on peut pas}\\
					\hline
					R & -8 & 0 & 0 & 0 & 20 & 1 & 10000
				\end{array}
			$$
			On retourne a l'�tape 2: \todo{ajouter le cercle autour du 1/30}
			$$
				\begin{array}{r|cccccc|c|l}
					Base & x & y & e_1 & e_2 & e_3 & R & Valeur & Quotient\\
					\hline
					x & 1 & 0 & \frac{1}{30} & 0 & -\frac{8}{3} & 0 & 666.\overline{66} & \text{n�gatif} \\
					e_2 & 0  & 0 & -\frac{1}{600} & 1 & \frac{1}{30} & 0 & \frac{20}{3} & \frac{\frac{20}{3}}{\frac{1}{30}} = 200\\
					y & 0 & 1 & 0 & 0 & 1 & 0 & 500 & \frac{500}{1}=500\\
					\hline
					R & 0 & 0 & \frac{4}{15} & 0 & -\frac{4}{3} & 1 & \frac{46000}{3}
				\end{array}
			$$
			$$
				\begin{array}{r|cccccc|c|l}
					Base & x & y & e_1 & e_2 & e_3 & R & Valeur & Quotient\\
					\hline
					x & 1 & 0 & -\frac{1}{10} & 80 & 0 & 0 & 1200\\
					e_3 & 0 & 0 & -\frac{1}{20} & 30 & 1 & 0 & 200\\
					y & 0 & 1 & \frac{1}{20} & -30 & 0 & 0 & 300\\
					\hline
					R & 0 & 0 & \frac{1}{5} & 40 & 0 & 1 & 15600
				\end{array}
			$$

			Par cons�quent, pour maximiser le profit, on doit produire $1200L$ de sirop et $300L$ de tire, le profit sera de $15600\$$
		\end{example}

		Si un probl�me de programmation lin�aire est exprim� sous une forme o� se retrouvent des contraintes d'in�galit�s, alors appara�t le couple primal-dual suivant:\\
		\todo{refaire l'alignement et s'assurer que le � du sujet � apparait correctement}\\
		\begin{itemize}
			\item Probl�me primal\\
				\begin{align*}{rrl}
					Maximiser & & \\
						& z &=c^tx \\
					Sujet � & & \\
						& Ax &\leq b\\
						& x &\geq 0\\
				\end{align*}
			\item Probl�me dual\\
				\begin{align*}{rrl}
					Minimizer & & \\
						& V & =b^ty \\
					Sujet � & & \\
						& A^ty & \geq c \\
						& y & \geq 0 
				\end{align*}
		\end{itemize}

		\todo{refaire l'alignement}\\
		De mani�re g�n�rale, on a les r�gles suivantes:
		\begin{enumerate}
			\item � toutes contraintes du probl�me primal correspond un variable du probl�me dual et � toute variable du probl�me primal correspond une contrainte du dual.

			\item � toute contrainte d'�galit� du probl�me primal correspond une variable libre du dual. Une variable $x_i$ rst libre su elle n'est pas contrainte � �tre n�gative.

			\item � une contrainte d'in�galit� du primal correspond une variable non n�gative ou non positive du probl�me dual.

			\item � une variable libre du primal correspond une contrainte d'�galit� du dual.

			\item � une variable non-n�gative ou non-positive du probl�me primal correspond une contrainte d'in�galit� du dual.
		\end{enumerate}

\backmatter

	
\end{document}